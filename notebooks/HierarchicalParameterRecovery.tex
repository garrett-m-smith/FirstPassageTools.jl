\documentclass[12pt,a4paper]{article}

\usepackage[a4paper,text={16.5cm,25.2cm},centering]{geometry}
\usepackage{lmodern}
\usepackage{amssymb,amsmath}
\usepackage{bm}
\usepackage{graphicx}
\usepackage{microtype}
\usepackage{hyperref}
\setlength{\parindent}{0pt}
\setlength{\parskip}{1.2ex}

\hypersetup
       {   pdfauthor = {  },
           pdftitle={  },
           colorlinks=TRUE,
           linkcolor=black,
           citecolor=blue,
           urlcolor=blue
       }




\usepackage{upquote}
\usepackage{listings}
\usepackage{xcolor}
\lstset{
    basicstyle=\ttfamily\footnotesize,
    upquote=true,
    breaklines=true,
    breakindent=0pt,
    keepspaces=true,
    showspaces=false,
    columns=fullflexible,
    showtabs=false,
    showstringspaces=false,
    escapeinside={(*@}{@*)},
    extendedchars=true,
}
\newcommand{\HLJLt}[1]{#1}
\newcommand{\HLJLw}[1]{#1}
\newcommand{\HLJLe}[1]{#1}
\newcommand{\HLJLeB}[1]{#1}
\newcommand{\HLJLo}[1]{#1}
\newcommand{\HLJLk}[1]{\textcolor[RGB]{148,91,176}{\textbf{#1}}}
\newcommand{\HLJLkc}[1]{\textcolor[RGB]{59,151,46}{\textit{#1}}}
\newcommand{\HLJLkd}[1]{\textcolor[RGB]{214,102,97}{\textit{#1}}}
\newcommand{\HLJLkn}[1]{\textcolor[RGB]{148,91,176}{\textbf{#1}}}
\newcommand{\HLJLkp}[1]{\textcolor[RGB]{148,91,176}{\textbf{#1}}}
\newcommand{\HLJLkr}[1]{\textcolor[RGB]{148,91,176}{\textbf{#1}}}
\newcommand{\HLJLkt}[1]{\textcolor[RGB]{148,91,176}{\textbf{#1}}}
\newcommand{\HLJLn}[1]{#1}
\newcommand{\HLJLna}[1]{#1}
\newcommand{\HLJLnb}[1]{#1}
\newcommand{\HLJLnbp}[1]{#1}
\newcommand{\HLJLnc}[1]{#1}
\newcommand{\HLJLncB}[1]{#1}
\newcommand{\HLJLnd}[1]{\textcolor[RGB]{214,102,97}{#1}}
\newcommand{\HLJLne}[1]{#1}
\newcommand{\HLJLneB}[1]{#1}
\newcommand{\HLJLnf}[1]{\textcolor[RGB]{66,102,213}{#1}}
\newcommand{\HLJLnfm}[1]{\textcolor[RGB]{66,102,213}{#1}}
\newcommand{\HLJLnp}[1]{#1}
\newcommand{\HLJLnl}[1]{#1}
\newcommand{\HLJLnn}[1]{#1}
\newcommand{\HLJLno}[1]{#1}
\newcommand{\HLJLnt}[1]{#1}
\newcommand{\HLJLnv}[1]{#1}
\newcommand{\HLJLnvc}[1]{#1}
\newcommand{\HLJLnvg}[1]{#1}
\newcommand{\HLJLnvi}[1]{#1}
\newcommand{\HLJLnvm}[1]{#1}
\newcommand{\HLJLl}[1]{#1}
\newcommand{\HLJLld}[1]{\textcolor[RGB]{148,91,176}{\textit{#1}}}
\newcommand{\HLJLs}[1]{\textcolor[RGB]{201,61,57}{#1}}
\newcommand{\HLJLsa}[1]{\textcolor[RGB]{201,61,57}{#1}}
\newcommand{\HLJLsb}[1]{\textcolor[RGB]{201,61,57}{#1}}
\newcommand{\HLJLsc}[1]{\textcolor[RGB]{201,61,57}{#1}}
\newcommand{\HLJLsd}[1]{\textcolor[RGB]{201,61,57}{#1}}
\newcommand{\HLJLsdB}[1]{\textcolor[RGB]{201,61,57}{#1}}
\newcommand{\HLJLsdC}[1]{\textcolor[RGB]{201,61,57}{#1}}
\newcommand{\HLJLse}[1]{\textcolor[RGB]{59,151,46}{#1}}
\newcommand{\HLJLsh}[1]{\textcolor[RGB]{201,61,57}{#1}}
\newcommand{\HLJLsi}[1]{#1}
\newcommand{\HLJLso}[1]{\textcolor[RGB]{201,61,57}{#1}}
\newcommand{\HLJLsr}[1]{\textcolor[RGB]{201,61,57}{#1}}
\newcommand{\HLJLss}[1]{\textcolor[RGB]{201,61,57}{#1}}
\newcommand{\HLJLssB}[1]{\textcolor[RGB]{201,61,57}{#1}}
\newcommand{\HLJLnB}[1]{\textcolor[RGB]{59,151,46}{#1}}
\newcommand{\HLJLnbB}[1]{\textcolor[RGB]{59,151,46}{#1}}
\newcommand{\HLJLnfB}[1]{\textcolor[RGB]{59,151,46}{#1}}
\newcommand{\HLJLnh}[1]{\textcolor[RGB]{59,151,46}{#1}}
\newcommand{\HLJLni}[1]{\textcolor[RGB]{59,151,46}{#1}}
\newcommand{\HLJLnil}[1]{\textcolor[RGB]{59,151,46}{#1}}
\newcommand{\HLJLnoB}[1]{\textcolor[RGB]{59,151,46}{#1}}
\newcommand{\HLJLoB}[1]{\textcolor[RGB]{102,102,102}{\textbf{#1}}}
\newcommand{\HLJLow}[1]{\textcolor[RGB]{102,102,102}{\textbf{#1}}}
\newcommand{\HLJLp}[1]{#1}
\newcommand{\HLJLc}[1]{\textcolor[RGB]{153,153,119}{\textit{#1}}}
\newcommand{\HLJLch}[1]{\textcolor[RGB]{153,153,119}{\textit{#1}}}
\newcommand{\HLJLcm}[1]{\textcolor[RGB]{153,153,119}{\textit{#1}}}
\newcommand{\HLJLcp}[1]{\textcolor[RGB]{153,153,119}{\textit{#1}}}
\newcommand{\HLJLcpB}[1]{\textcolor[RGB]{153,153,119}{\textit{#1}}}
\newcommand{\HLJLcs}[1]{\textcolor[RGB]{153,153,119}{\textit{#1}}}
\newcommand{\HLJLcsB}[1]{\textcolor[RGB]{153,153,119}{\textit{#1}}}
\newcommand{\HLJLg}[1]{#1}
\newcommand{\HLJLgd}[1]{#1}
\newcommand{\HLJLge}[1]{#1}
\newcommand{\HLJLgeB}[1]{#1}
\newcommand{\HLJLgh}[1]{#1}
\newcommand{\HLJLgi}[1]{#1}
\newcommand{\HLJLgo}[1]{#1}
\newcommand{\HLJLgp}[1]{#1}
\newcommand{\HLJLgs}[1]{#1}
\newcommand{\HLJLgsB}[1]{#1}
\newcommand{\HLJLgt}[1]{#1}


\begin{document}



\rule{\textwidth}{1pt}
author: Garrett Smith title: Parameter recovery with a hierarchical model \ensuremath{\endash}-

\section{Parameter recovery test using a hierarchical model}
The goal of this script is to test the recovery of parameters from a hierarchical model


\begin{lstlisting}
(*@\HLJLcs{{\#}with}@*) (*@\HLJLcs{a}@*) (*@\HLJLcs{{\textasciigrave}fpdistribution{\textasciigrave}}@*) (*@\HLJLcs{likelihood.}@*)

(*@\HLJLk{using}@*) (*@\HLJLn{Distributions}@*)
(*@\HLJLk{using}@*) (*@\HLJLn{Turing}@*)
(*@\HLJLk{using}@*) (*@\HLJLn{Plots}@*)(*@\HLJLp{,}@*) (*@\HLJLn{StatsPlots}@*)
(*@\HLJLk{using}@*) (*@\HLJLn{Pkg}@*)
(*@\HLJLn{Pkg}@*)(*@\HLJLoB{.}@*)(*@\HLJLnf{activate}@*)(*@\HLJLp{(}@*)(*@\HLJLs{"{}../../FirstPassageTools.jl/"{}}@*)(*@\HLJLp{)}@*)
(*@\HLJLk{using}@*) (*@\HLJLn{FirstPassageTools}@*)
\end{lstlisting}


\subsection{Generating the true data}
First, we set up the transition rate matrices for the first-passage time distribution we want to fit.


\begin{lstlisting}
(*@\HLJLn{T}@*) (*@\HLJLoB{=}@*) (*@\HLJLni{4}@*)(*@\HLJLoB{*}@*)(*@\HLJLp{[}@*)(*@\HLJLoB{-}@*)(*@\HLJLni{1}@*) (*@\HLJLni{0}@*) (*@\HLJLni{0}@*)(*@\HLJLp{;}@*) (*@\HLJLni{1}@*) (*@\HLJLoB{-}@*)(*@\HLJLni{1}@*) (*@\HLJLni{1}@*)(*@\HLJLp{;}@*) (*@\HLJLni{0}@*) (*@\HLJLni{1}@*) (*@\HLJLoB{-}@*)(*@\HLJLni{2}@*)(*@\HLJLp{]}@*)
(*@\HLJLn{A}@*) (*@\HLJLoB{=}@*) (*@\HLJLni{4}@*)(*@\HLJLoB{*}@*)(*@\HLJLp{[}@*)(*@\HLJLni{0}@*) (*@\HLJLni{0}@*) (*@\HLJLni{1}@*)(*@\HLJLp{]}@*)
(*@\HLJLn{p0}@*) (*@\HLJLoB{=}@*) (*@\HLJLp{[}@*)(*@\HLJLnfB{1.0}@*)(*@\HLJLp{,}@*) (*@\HLJLni{0}@*)(*@\HLJLp{,}@*) (*@\HLJLni{0}@*)(*@\HLJLp{]}@*)
\end{lstlisting}

\begin{lstlisting}
3-element Vector(*@{{\{}}@*)Float64(*@{{\}}}@*):
 1.0
 0.0
 0.0
\end{lstlisting}


Scaling the transition rate matrices by \texttt{\ensuremath{\tau} = 2.5} should give mean first-passage times of around 400ms. Generating and fitting the paramters (\ensuremath{\tau} and the separate \ensuremath{\tau}\ensuremath{\_i}) will be done on the log scale and then exponentiated in order to keep the transition rates positive.


\begin{lstlisting}
(*@\HLJLn{nparticipants}@*) (*@\HLJLoB{=}@*) (*@\HLJLni{20}@*)
(*@\HLJLn{true{\_}tau}@*) (*@\HLJLoB{=}@*) (*@\HLJLnf{log}@*)(*@\HLJLp{(}@*)(*@\HLJLnfB{2.5}@*)(*@\HLJLp{)}@*)
(*@\HLJLn{true{\_}sd}@*) (*@\HLJLoB{=}@*) (*@\HLJLnfB{0.25}@*)
(*@\HLJLn{true{\_}tau{\_}i}@*) (*@\HLJLoB{=}@*) (*@\HLJLnf{rand}@*)(*@\HLJLp{(}@*)(*@\HLJLnf{Normal}@*)(*@\HLJLp{(}@*)(*@\HLJLni{0}@*)(*@\HLJLp{,}@*) (*@\HLJLn{true{\_}sd}@*)(*@\HLJLp{),}@*) (*@\HLJLn{nparticipants}@*)(*@\HLJLp{)}@*)
\end{lstlisting}

\begin{lstlisting}
20-element Vector(*@{{\{}}@*)Float64(*@{{\}}}@*):
 -0.3320002860864489
  0.06049712955896572
 -0.039792049491069785
 -0.24596052788083897
  0.3074023245459109
  0.08219836702929782
 -0.09364450976323194
 -0.04604100307585175
 -0.050116629273839385
  0.06685945716140113
 -0.1076187874617376
  0.19537766599138914
 -0.4897451838973754
 -0.31289022358878843
 -0.20250100906866791
 -0.5486531423909452
 -0.34106867087000914
  0.1742495612824492
 -0.02476713336699382
 -0.20324356282205475
\end{lstlisting}


The data will be saved in wide format: Each participant's data corresponds to a row, and each column is a data point.


\begin{lstlisting}
(*@\HLJLn{ndata}@*) (*@\HLJLoB{=}@*) (*@\HLJLni{20}@*)
(*@\HLJLn{data}@*) (*@\HLJLoB{=}@*) (*@\HLJLnf{zeros}@*)(*@\HLJLp{(}@*)(*@\HLJLn{nparticipants}@*)(*@\HLJLp{,}@*) (*@\HLJLn{ndata}@*)(*@\HLJLp{)}@*)
(*@\HLJLn{param}@*) (*@\HLJLoB{=}@*) (*@\HLJLn{exp}@*)(*@\HLJLoB{.}@*)(*@\HLJLp{(}@*)(*@\HLJLn{true{\_}tau}@*) (*@\HLJLoB{.+}@*) (*@\HLJLn{true{\_}tau{\_}i}@*)(*@\HLJLp{)}@*)
(*@\HLJLk{for}@*) (*@\HLJLn{i}@*) (*@\HLJLoB{=}@*) (*@\HLJLni{1}@*)(*@\HLJLoB{:}@*)(*@\HLJLn{nparticipants}@*)
    (*@\HLJLn{data}@*)(*@\HLJLp{[}@*)(*@\HLJLn{i}@*)(*@\HLJLp{,}@*)(*@\HLJLoB{:}@*)(*@\HLJLp{]}@*) (*@\HLJLoB{=}@*) (*@\HLJLnf{rand}@*)(*@\HLJLp{(}@*)(*@\HLJLnf{fpdistribution}@*)(*@\HLJLp{(}@*)(*@\HLJLn{param}@*)(*@\HLJLp{[}@*)(*@\HLJLn{i}@*)(*@\HLJLp{]}@*)(*@\HLJLoB{*}@*)(*@\HLJLn{T}@*)(*@\HLJLp{,}@*) (*@\HLJLn{param}@*)(*@\HLJLp{[}@*)(*@\HLJLn{i}@*)(*@\HLJLp{]}@*)(*@\HLJLoB{*}@*)(*@\HLJLn{A}@*)(*@\HLJLp{,}@*) (*@\HLJLn{p0}@*)(*@\HLJLp{),}@*) (*@\HLJLn{ndata}@*)(*@\HLJLp{)}@*)
(*@\HLJLk{end}@*)
\end{lstlisting}


\subsection{Specifying the model}
First, we specify the priors on the log scale:


\begin{lstlisting}
(*@\HLJLn{pr{\_}tau}@*) (*@\HLJLoB{=}@*) (*@\HLJLnf{Normal}@*)(*@\HLJLp{(}@*)(*@\HLJLni{1}@*)(*@\HLJLp{,}@*) (*@\HLJLnfB{0.25}@*)(*@\HLJLp{)}@*)
(*@\HLJLn{pr{\_}sd}@*) (*@\HLJLoB{=}@*) (*@\HLJLnf{Exponential}@*)(*@\HLJLp{(}@*)(*@\HLJLnfB{0.5}@*)(*@\HLJLp{)}@*)  (*@\HLJLcs{{\#}}@*) (*@\HLJLcs{Prior}@*) (*@\HLJLcs{on}@*) (*@\HLJLcs{the}@*) (*@\HLJLcs{SD}@*) (*@\HLJLcs{of}@*) (*@\HLJLcs{the}@*) (*@\HLJLcs{\ensuremath{\tau}\ensuremath{\_i}}@*)
\end{lstlisting}

\begin{lstlisting}
Distributions.Exponential(*@{{\{}}@*)Float64(*@{{\}}}@*)((*@\ensuremath{\theta}@*)=0.5)
\end{lstlisting}


Next, we can write the full model including the likelihood


\begin{lstlisting}
(*@\HLJLnd{@model}@*) (*@\HLJLk{function}@*) (*@\HLJLnf{mod}@*)(*@\HLJLp{(}@*)(*@\HLJLn{y}@*)(*@\HLJLp{)}@*)
    (*@\HLJLn{np}@*) (*@\HLJLoB{=}@*) (*@\HLJLnf{size}@*)(*@\HLJLp{(}@*)(*@\HLJLn{y}@*)(*@\HLJLp{,}@*) (*@\HLJLni{1}@*)(*@\HLJLp{)}@*)
    (*@\HLJLn{nd}@*) (*@\HLJLoB{=}@*) (*@\HLJLnf{size}@*)(*@\HLJLp{(}@*)(*@\HLJLn{y}@*)(*@\HLJLp{,}@*) (*@\HLJLni{2}@*)(*@\HLJLp{)}@*)
    (*@\HLJLcs{{\#}}@*) (*@\HLJLcs{Priors}@*)
    (*@\HLJLn{\ensuremath{\tau}}@*) (*@\HLJLoB{{\textasciitilde}}@*) (*@\HLJLn{pr{\_}tau}@*)
    (*@\HLJLcs{{\#}}@*) (*@\HLJLcs{Need}@*) (*@\HLJLcs{to}@*) (*@\HLJLcs{initialize}@*) (*@\HLJLcs{as}@*) (*@\HLJLcs{a}@*) (*@\HLJLcs{TArray}@*) (*@\HLJLcs{b/c}@*) (*@\HLJLcs{PG}@*) (*@\HLJLcs{sampler}@*)
    (*@\HLJLn{\ensuremath{\tau}\ensuremath{\_i}}@*) (*@\HLJLoB{=}@*) (*@\HLJLnf{tzeros}@*)(*@\HLJLp{(}@*)(*@\HLJLn{Float64}@*)(*@\HLJLp{,}@*) (*@\HLJLn{np}@*)(*@\HLJLp{)}@*)
    (*@\HLJLn{sd}@*) (*@\HLJLoB{{\textasciitilde}}@*) (*@\HLJLn{pr{\_}sd}@*)
    (*@\HLJLn{\ensuremath{\tau}\ensuremath{\_i}}@*) (*@\HLJLoB{{\textasciitilde}}@*) (*@\HLJLnf{filldist}@*)(*@\HLJLp{(}@*)(*@\HLJLnf{Normal}@*)(*@\HLJLp{(}@*)(*@\HLJLni{0}@*)(*@\HLJLp{,}@*) (*@\HLJLn{sd}@*)(*@\HLJLp{),}@*) (*@\HLJLn{np}@*)(*@\HLJLp{)}@*)

    (*@\HLJLcs{{\#}}@*) (*@\HLJLcs{Likelihood}@*)
    (*@\HLJLn{mult}@*) (*@\HLJLoB{=}@*) (*@\HLJLn{exp}@*)(*@\HLJLoB{.}@*)(*@\HLJLp{(}@*)(*@\HLJLn{\ensuremath{\tau}}@*) (*@\HLJLoB{.+}@*) (*@\HLJLn{\ensuremath{\tau}\ensuremath{\_i}}@*)(*@\HLJLp{)}@*)
    (*@\HLJLn{y}@*) (*@\HLJLoB{{\textasciitilde}}@*) (*@\HLJLnf{filldist}@*)(*@\HLJLp{(}@*)(*@\HLJLnf{arraydist}@*)(*@\HLJLp{([}@*)(*@\HLJLnf{fpdistribution}@*)(*@\HLJLp{(}@*)(*@\HLJLn{mult}@*)(*@\HLJLp{[}@*)(*@\HLJLn{p}@*)(*@\HLJLp{]}@*)(*@\HLJLoB{*}@*)(*@\HLJLn{T}@*)(*@\HLJLp{,}@*) (*@\HLJLn{mult}@*)(*@\HLJLp{[}@*)(*@\HLJLn{p}@*)(*@\HLJLp{]}@*)(*@\HLJLoB{*}@*)(*@\HLJLn{A}@*)(*@\HLJLp{,}@*) (*@\HLJLn{p0}@*)(*@\HLJLp{)}@*) (*@\HLJLk{for}@*) (*@\HLJLn{p}@*) (*@\HLJLkp{in}@*) (*@\HLJLni{1}@*)(*@\HLJLoB{:}@*)(*@\HLJLn{np}@*)(*@\HLJLp{]),}@*) (*@\HLJLn{nd}@*)(*@\HLJLp{)}@*)
(*@\HLJLk{end}@*)
\end{lstlisting}

\begin{lstlisting}
mod (generic function with 2 methods)
\end{lstlisting}


\subsection{Sampling}
Here, we'll use the particle Gibbs sampler with 50 particles to sample from the posterior. We'll use four chains of 1000 samples each. Make sure to execute this script with \texttt{julia -t 4 HierarchicalParameterRecovery.jl}.


\begin{lstlisting}
(*@\HLJLn{posterior}@*) (*@\HLJLoB{=}@*) (*@\HLJLnf{sample}@*)(*@\HLJLp{(}@*)(*@\HLJLnf{mod}@*)(*@\HLJLp{(}@*)(*@\HLJLn{data}@*)(*@\HLJLp{),}@*) (*@\HLJLnf{PG}@*)(*@\HLJLp{(}@*)(*@\HLJLni{50}@*)(*@\HLJLp{),}@*) (*@\HLJLnf{MCMCThreads}@*)(*@\HLJLp{(),}@*) (*@\HLJLni{1000}@*)(*@\HLJLp{,}@*) (*@\HLJLni{4}@*)(*@\HLJLp{)}@*)
\end{lstlisting}

\begin{lstlisting}
Chains MCMC chain (1000(*@\ensuremath{\times}@*)24(*@\ensuremath{\times}@*)4 Array(*@{{\{}}@*)Float64, 3(*@{{\}}}@*)):

Iterations        = 1:1:1000
Number of chains  = 4
Samples per chain = 1000
Wall duration     = 334.57 seconds
Compute duration  = 1332.88 seconds
parameters        = (*@\ensuremath{\tau}@*), sd, (*@\ensuremath{\tau}@*)(*@\ensuremath{\_i}@*)[1], (*@\ensuremath{\tau}@*)(*@\ensuremath{\_i}@*)[2], (*@\ensuremath{\tau}@*)(*@\ensuremath{\_i}@*)[3], (*@\ensuremath{\tau}@*)(*@\ensuremath{\_i}@*)[4], (*@\ensuremath{\tau}@*)(*@\ensuremath{\_i}@*)[5], (*@\ensuremath{\tau}@*)(*@\ensuremath{\_i}@*)[6], (*@\ensuremath{\tau}@*)(*@\ensuremath{\_i}@*)[7],
 (*@\ensuremath{\tau}@*)(*@\ensuremath{\_i}@*)[8], (*@\ensuremath{\tau}@*)(*@\ensuremath{\_i}@*)[9], (*@\ensuremath{\tau}@*)(*@\ensuremath{\_i}@*)[10], (*@\ensuremath{\tau}@*)(*@\ensuremath{\_i}@*)[11], (*@\ensuremath{\tau}@*)(*@\ensuremath{\_i}@*)[12], (*@\ensuremath{\tau}@*)(*@\ensuremath{\_i}@*)[13], (*@\ensuremath{\tau}@*)(*@\ensuremath{\_i}@*)[14], (*@\ensuremath{\tau}@*)(*@\ensuremath{\_i}@*)[15], (*@\ensuremath{\tau}@*)(*@\ensuremath{\_i}@*)[16], (*@\ensuremath{\tau}@*)(*@\ensuremath{\_i}@*)[1
7], (*@\ensuremath{\tau}@*)(*@\ensuremath{\_i}@*)[18], (*@\ensuremath{\tau}@*)(*@\ensuremath{\_i}@*)[19], (*@\ensuremath{\tau}@*)(*@\ensuremath{\_i}@*)[20]
internals         = lp, logevidence

Summary Statistics
  parameters      mean       std   naive(*@{{\_}}@*)se      mcse       ess      rhat  
 es (*@\ensuremath{\cdots}@*)
      Symbol   Float64   Float64    Float64   Float64   Float64   Float64  
    (*@\ensuremath{\cdots}@*)

           (*@\ensuremath{\tau}@*)    0.7877    0.0488     0.0008    0.0055   13.7494    1.4641  
    (*@\ensuremath{\cdots}@*)
          sd    0.1724    0.0634     0.0010    0.0071   10.7189    1.9442  
    (*@\ensuremath{\cdots}@*)
       (*@\ensuremath{\tau}@*)(*@\ensuremath{\_i}@*)[1]   -0.0402    0.1169     0.0018    0.0129   12.9036    1.5620  
    (*@\ensuremath{\cdots}@*)
       (*@\ensuremath{\tau}@*)(*@\ensuremath{\_i}@*)[2]    0.1039    0.1376     0.0022    0.0156   20.4472    1.2521  
    (*@\ensuremath{\cdots}@*)
       (*@\ensuremath{\tau}@*)(*@\ensuremath{\_i}@*)[3]   -0.0018    0.1101     0.0017    0.0130   20.0191    1.1945  
    (*@\ensuremath{\cdots}@*)
       (*@\ensuremath{\tau}@*)(*@\ensuremath{\_i}@*)[4]   -0.0304    0.1682     0.0027    0.0205    9.7269    2.2513  
    (*@\ensuremath{\cdots}@*)
       (*@\ensuremath{\tau}@*)(*@\ensuremath{\_i}@*)[5]    0.1770    0.1269     0.0020    0.0143   21.5904    1.1908  
    (*@\ensuremath{\cdots}@*)
       (*@\ensuremath{\tau}@*)(*@\ensuremath{\_i}@*)[6]    0.0351    0.0944     0.0015    0.0107   16.3700    1.3609  
    (*@\ensuremath{\cdots}@*)
       (*@\ensuremath{\tau}@*)(*@\ensuremath{\_i}@*)[7]   -0.0186    0.1145     0.0018    0.0137   11.6923    1.7436  
    (*@\ensuremath{\cdots}@*)
       (*@\ensuremath{\tau}@*)(*@\ensuremath{\_i}@*)[8]   -0.0131    0.1101     0.0017    0.0127   15.2447    1.3733  
    (*@\ensuremath{\cdots}@*)
       (*@\ensuremath{\tau}@*)(*@\ensuremath{\_i}@*)[9]    0.0149    0.1496     0.0024    0.0169   16.3648    1.3508  
    (*@\ensuremath{\cdots}@*)
      (*@\ensuremath{\tau}@*)(*@\ensuremath{\_i}@*)[10]    0.1266    0.0818     0.0013    0.0083   27.5361    1.2007  
    (*@\ensuremath{\cdots}@*)
      (*@\ensuremath{\tau}@*)(*@\ensuremath{\_i}@*)[11]    0.0507    0.0937     0.0015    0.0100   24.1764    1.2104  
    (*@\ensuremath{\cdots}@*)
      (*@\ensuremath{\tau}@*)(*@\ensuremath{\_i}@*)[12]   -0.0150    0.1569     0.0025    0.0184   11.4058    1.7623  
    (*@\ensuremath{\cdots}@*)
      (*@\ensuremath{\tau}@*)(*@\ensuremath{\_i}@*)[13]   -0.2765    0.1689     0.0027    0.0196   10.7669    1.9500  
    (*@\ensuremath{\cdots}@*)
      (*@\ensuremath{\tau}@*)(*@\ensuremath{\_i}@*)[14]   -0.0516    0.0965     0.0015    0.0110   16.0604    1.3410  
    (*@\ensuremath{\cdots}@*)
      (*@\ensuremath{\tau}@*)(*@\ensuremath{\_i}@*)[15]   -0.0303    0.1518     0.0024    0.0181   13.6745    1.4610  
    (*@\ensuremath{\cdots}@*)
      (*@\ensuremath{\vdots}@*)           (*@\ensuremath{\vdots}@*)         (*@\ensuremath{\vdots}@*)         (*@\ensuremath{\vdots}@*)          (*@\ensuremath{\vdots}@*)         (*@\ensuremath{\vdots}@*)         (*@\ensuremath{\vdots}@*)     
    (*@\ensuremath{\ddots}@*)
                                                     1 column and 5 rows om
itted

Quantiles
  parameters      2.5(*@{{\%}}@*)     25.0(*@{{\%}}@*)     50.0(*@{{\%}}@*)     75.0(*@{{\%}}@*)     97.5(*@{{\%}}@*)
      Symbol   Float64   Float64   Float64   Float64   Float64

           (*@\ensuremath{\tau}@*)    0.7030    0.7567    0.7993    0.8327    0.8397
          sd    0.0954    0.1383    0.1494    0.2300    0.2839
       (*@\ensuremath{\tau}@*)(*@\ensuremath{\_i}@*)[1]   -0.2113   -0.0989   -0.0465    0.0770    0.1920
       (*@\ensuremath{\tau}@*)(*@\ensuremath{\_i}@*)[2]   -0.1794    0.0366    0.1237    0.2344    0.3144
       (*@\ensuremath{\tau}@*)(*@\ensuremath{\_i}@*)[3]   -0.2194   -0.0570   -0.0426    0.1093    0.1839
       (*@\ensuremath{\tau}@*)(*@\ensuremath{\_i}@*)[4]   -0.2748   -0.1325   -0.0598    0.0893    0.2484
       (*@\ensuremath{\tau}@*)(*@\ensuremath{\_i}@*)[5]   -0.1093    0.1383    0.1532    0.1738    0.4578
       (*@\ensuremath{\tau}@*)(*@\ensuremath{\_i}@*)[6]   -0.0975   -0.0634    0.0385    0.0940    0.3075
       (*@\ensuremath{\tau}@*)(*@\ensuremath{\_i}@*)[7]   -0.3342   -0.0556   -0.0207    0.0553    0.2146
       (*@\ensuremath{\tau}@*)(*@\ensuremath{\_i}@*)[8]   -0.2371   -0.0809   -0.0658    0.0538    0.2010
       (*@\ensuremath{\tau}@*)(*@\ensuremath{\_i}@*)[9]   -0.2528   -0.0701   -0.0255    0.1193    0.3291
      (*@\ensuremath{\tau}@*)(*@\ensuremath{\_i}@*)[10]   -0.0312    0.0734    0.1351    0.1868    0.3215
      (*@\ensuremath{\tau}@*)(*@\ensuremath{\_i}@*)[11]   -0.1771    0.0318    0.0540    0.1142    0.1765
      (*@\ensuremath{\tau}@*)(*@\ensuremath{\_i}@*)[12]   -0.2826   -0.0718   -0.0548    0.1383    0.2414
      (*@\ensuremath{\tau}@*)(*@\ensuremath{\_i}@*)[13]   -0.6076   -0.3980   -0.2614   -0.1551   -0.0244
      (*@\ensuremath{\tau}@*)(*@\ensuremath{\_i}@*)[14]   -0.2740   -0.1270   -0.0390    0.0115    0.1120
      (*@\ensuremath{\tau}@*)(*@\ensuremath{\_i}@*)[15]   -0.2420   -0.1645    0.0163    0.1141    0.2069
      (*@\ensuremath{\vdots}@*)           (*@\ensuremath{\vdots}@*)         (*@\ensuremath{\vdots}@*)         (*@\ensuremath{\vdots}@*)         (*@\ensuremath{\vdots}@*)         (*@\ensuremath{\vdots}@*)
                                                  5 rows omitted
\end{lstlisting}


\subsection{Evaluating parameter recovery}
First, we summarize the chains:


\begin{lstlisting}
(*@\HLJLnf{describe}@*)(*@\HLJLp{(}@*)(*@\HLJLn{posterior}@*)(*@\HLJLp{)}@*)
\end{lstlisting}

\begin{lstlisting}
2-element Vector(*@{{\{}}@*)MCMCChains.ChainDataFrame(*@{{\}}}@*):
 Summary Statistics (22 x 8)
 Quantiles (22 x 6)
\end{lstlisting}


And plot them:


\begin{lstlisting}
(*@\HLJLnf{histogram}@*)(*@\HLJLp{(}@*)(*@\HLJLn{posterior}@*)(*@\HLJLp{[}@*)(*@\HLJLsc{:\ensuremath{\tau}}@*)(*@\HLJLp{][}@*)(*@\HLJLoB{:}@*)(*@\HLJLp{],}@*) (*@\HLJLn{xlabel}@*)(*@\HLJLoB{=}@*)(*@\HLJLs{"{}\ensuremath{\tau}"{}}@*)(*@\HLJLp{)}@*)
(*@\HLJLnf{vline!}@*)(*@\HLJLp{([}@*)(*@\HLJLn{true{\_}tau}@*)(*@\HLJLp{],}@*) (*@\HLJLn{label}@*)(*@\HLJLoB{=}@*)(*@\HLJLs{"{}True}@*) (*@\HLJLs{value"{}}@*)(*@\HLJLp{)}@*)
(*@\HLJLnf{savefig}@*)(*@\HLJLp{(}@*)(*@\HLJLs{"{}tau{\_}posterior.pdf"{}}@*)(*@\HLJLp{)}@*)

(*@\HLJLnf{histogram}@*)(*@\HLJLp{(}@*)(*@\HLJLn{posterior}@*)(*@\HLJLp{[}@*)(*@\HLJLsc{:sd}@*)(*@\HLJLp{][}@*)(*@\HLJLoB{:}@*)(*@\HLJLp{],}@*) (*@\HLJLn{xlabel}@*)(*@\HLJLoB{=}@*)(*@\HLJLs{"{}Std.}@*) (*@\HLJLs{deviation"{}}@*)(*@\HLJLp{)}@*)
(*@\HLJLnf{vline!}@*)(*@\HLJLp{([}@*)(*@\HLJLn{true{\_}sd}@*)(*@\HLJLp{],}@*) (*@\HLJLn{label}@*)(*@\HLJLoB{=}@*)(*@\HLJLs{"{}True}@*) (*@\HLJLs{value"{}}@*)(*@\HLJLp{)}@*)
(*@\HLJLnf{savefig}@*)(*@\HLJLp{(}@*)(*@\HLJLs{"{}sd{\_}posterior.pdf"{}}@*)(*@\HLJLp{)}@*)

(*@\HLJLn{plot{\_}vec}@*) (*@\HLJLoB{=}@*) (*@\HLJLp{[]}@*)
(*@\HLJLk{for}@*) (*@\HLJLn{p}@*) (*@\HLJLoB{=}@*) (*@\HLJLni{1}@*)(*@\HLJLoB{:}@*)(*@\HLJLn{nparticipants}@*)
    (*@\HLJLn{curr}@*) (*@\HLJLoB{=}@*) (*@\HLJLn{posterior}@*)(*@\HLJLoB{.}@*)(*@\HLJLn{name{\_}map}@*)(*@\HLJLoB{.}@*)(*@\HLJLn{parameters}@*)(*@\HLJLp{[}@*)(*@\HLJLn{p}@*)(*@\HLJLp{]}@*)
    (*@\HLJLn{plt}@*) (*@\HLJLoB{=}@*) (*@\HLJLnf{histogram}@*)(*@\HLJLp{(}@*)(*@\HLJLn{posterior}@*)(*@\HLJLp{[}@*)(*@\HLJLn{curr}@*)(*@\HLJLp{][}@*)(*@\HLJLoB{:}@*)(*@\HLJLp{])}@*)
    (*@\HLJLn{plt}@*) (*@\HLJLoB{=}@*) (*@\HLJLnf{vline!}@*)(*@\HLJLp{(}@*)(*@\HLJLn{plt}@*)(*@\HLJLp{,}@*) (*@\HLJLp{[}@*)(*@\HLJLn{true{\_}tau{\_}i}@*)(*@\HLJLp{[}@*)(*@\HLJLn{p}@*)(*@\HLJLp{]])}@*)
    (*@\HLJLnf{push!}@*)(*@\HLJLp{(}@*)(*@\HLJLn{plot{\_}vec}@*)(*@\HLJLp{,}@*) (*@\HLJLn{plt}@*)(*@\HLJLp{)}@*)
(*@\HLJLk{end}@*)
(*@\HLJLnf{plot}@*)(*@\HLJLp{(}@*)(*@\HLJLn{plot{\_}vec}@*)(*@\HLJLoB{...}@*)(*@\HLJLp{,}@*) (*@\HLJLn{legend}@*)(*@\HLJLoB{=}@*)(*@\HLJLkc{false}@*)(*@\HLJLp{,}@*) (*@\HLJLn{link}@*)(*@\HLJLoB{=:}@*)(*@\HLJLn{all}@*)(*@\HLJLp{)}@*)
(*@\HLJLnf{savefig}@*)(*@\HLJLp{(}@*)(*@\HLJLs{"{}tau{\_}i{\_}posterior.pdf"{}}@*)(*@\HLJLp{)}@*)
\end{lstlisting}


If the posterior contains the true values of the parameters, we can say the parameters were recovered successfully.

Let's also look at the Gelman-Rubin statistic for the chains:


\begin{lstlisting}
(*@\HLJLnf{gelmandiag}@*)(*@\HLJLp{(}@*)(*@\HLJLn{posterior}@*)(*@\HLJLp{)}@*)
\end{lstlisting}

\begin{lstlisting}
Gelman, Rubin, and Brooks diagnostic
  parameters      psrf    psrfci
      Symbol   Float64   Float64

           (*@\ensuremath{\tau}@*)    1.6244    2.4227
          sd    1.5032    2.7904
       (*@\ensuremath{\tau}@*)(*@\ensuremath{\_i}@*)[1]    1.6373    2.6713
       (*@\ensuremath{\tau}@*)(*@\ensuremath{\_i}@*)[2]    1.4710    2.6033
       (*@\ensuremath{\tau}@*)(*@\ensuremath{\_i}@*)[3]    1.1636    1.4788
       (*@\ensuremath{\tau}@*)(*@\ensuremath{\_i}@*)[4]    2.6537    4.6266
       (*@\ensuremath{\tau}@*)(*@\ensuremath{\_i}@*)[5]    1.3239    2.3018
       (*@\ensuremath{\tau}@*)(*@\ensuremath{\_i}@*)[6]    1.5911    2.5098
       (*@\ensuremath{\tau}@*)(*@\ensuremath{\_i}@*)[7]    1.2741    1.9523
       (*@\ensuremath{\tau}@*)(*@\ensuremath{\_i}@*)[8]    1.2754    1.7586
       (*@\ensuremath{\tau}@*)(*@\ensuremath{\_i}@*)[9]    1.2316    1.7391
      (*@\ensuremath{\tau}@*)(*@\ensuremath{\_i}@*)[10]    1.0461    1.1256
      (*@\ensuremath{\tau}@*)(*@\ensuremath{\_i}@*)[11]    1.1414    1.3584
      (*@\ensuremath{\tau}@*)(*@\ensuremath{\_i}@*)[12]    1.5741    2.4085
      (*@\ensuremath{\tau}@*)(*@\ensuremath{\_i}@*)[13]    1.4588    2.4631
      (*@\ensuremath{\tau}@*)(*@\ensuremath{\_i}@*)[14]    1.1355    1.3609
      (*@\ensuremath{\tau}@*)(*@\ensuremath{\_i}@*)[15]    1.6632    2.8722
      (*@\ensuremath{\vdots}@*)           (*@\ensuremath{\vdots}@*)         (*@\ensuremath{\vdots}@*)
                    5 rows omitted
\end{lstlisting}



\end{document}
